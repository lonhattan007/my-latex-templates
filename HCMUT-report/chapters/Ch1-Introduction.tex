%! Compiler = lualatex --shell-escape
%! BibTeX Compiler = biber

\documentclass[../main.tex]{subfiles}
\graphicspath{{\subfix{../figures/}}}

\begin{document}

	\justifying


	\chapter{Giới thiệu đề tài}
	\label{chap:intro}


	\section{Bối cảnh}

	Hiện nay, du lịch Đà Lạt ngaày càng phát triển; tuy nhiên khách du lịch thường gặp khó khăn trong việc thuê xe máy
	để tự di chuyển. Hầu hết các địa điểm cho thuê xe đều được xây dựng theo hình thức kinh doanh nhỏ lẻ, không có sự
	đảm bảo về chất lượng của xe cũng như giá thành. Hơn nữa, trong các mùa cao điểm du lịch thì việc tìm kiếm xe càng
	trở nên khó khăn. Về phía người cho thuê xe, một số chủ xe muốn cho thuê xe nhưng lại không có đủ năng lực để vận
	hành một đại lý hoặc khó quản lý hoạt động thuê xe. Vì vậy, việc xây dựng một ứng dụng giúp khách du lịch dễ dàng
	đặt xe cũng như kiểm tra chất lượng xe với giá cả minh bạch hợp lý cũng như hỗ trợ các chủ xe lẻ hoặc các đơn vị
	cho thuê xe gặp được tệp khách hàng và quản lý việc thuê xe là điều cần thiết nhằm phất triển du lịch Đà Lạt; sau
	đó có thể mở rộng quy mô ra các địa điểm khác trong và ngoài nước.


	\section{Tìm hiểu các ứng dụng, giải pháp có trên thị trường}

	Để có thể hình dung được rõ hơn về hệ thống sẽ xây dụng, nhóm tác giả đã tham khảo một số hệ thống tương tự trong
	lĩnh vực thuê xe đã có trên thị trường để phân tích được những ưu điểm và hạn chế của các ứng dụng đó. Trên thị
	trường hiện nay, có nhiều hệ thống hỗ trợ người dùng thuê xe. Đó có thể là hệ thống đặt thuê xe online của một đại
	lý chuyên cho thuê xe, cũng có thể là một phần tính năng của một hệ thống lớn hơn chuyên hỗ trợ khách du lịch,
	hoặc cũng có thể là một hệ thống trung gian giống với sản phẩm mà đề tài hướng tới hiện thực.

	Với nhóm ứng dụng thứ nhất, đây là các hệ thống mà các đại lý thuê xe lớn dùng để tiếp cận tới nhiều khách hàng
	hơn cũng như quản lý được việc cho thuê xe dễ dàng hơn. Các hệ thống này thường sẽ có giao diện web để khách thuê
	xe có thể truy cập và lựa chọn thuê xe theo kế hoạch cá nhân, đồng thời cung cấp một kênh liên lạc (như Messenger,
	Zalo \ldots) để nhân viên đại lý có thể hỗ trợ khách hàng. Đa phần các hệ thống này không có ứng dụng riêng cho các
	thiết bị điện thoại thông minh và có flow sử dụng đơn giản. Các hệ thống này đáp ứng được nhu cầu thuê xe tự lái cơ
	bản, nhưng không thể đáp ứng nhu cầu các chủ xe lẻ có nhu cầu cho thuê, hoặc một số trải nghiệm nâng cao hơn mà có
	thể khách thuê xe cần.

	Tiếp theo là nhóm ứng dụng thứ hai - các ứng dụng du lịch lớn mà khách hàng có thể đặt một số dịch vụ du lịch khác,
	bao gồm thuê xe tự lái. Một số ứng dụng lớn có thể kể tên tới như \emph{
		TripAdvisor, Viator, Traveloka, Booking .com, Trip.com \ldots} Vì đây là các nền tảng lớn nên giao diện và lu
	ồng sử dụng của chúng được thiết kế tốt, trên cả nền tảng web và mobile. Tuy nhiên khảo sát cho thấy không có
	nhiều dịch vụ cho thuê xe máy trên các ứng dụng này. Các chủ xe không bị hạn chế trong việc chủ động cho đăng ký
	thuê xe.

	Cuối cùng là nhóm ứng dụng sàn thuê xe trung gian. Chúng đóng vai trò như một nền tảng thương mại điện tử, kết nối
	khách hàng có nhu cầu thuê xe ô tô tự lái hoặc xe có tài xế với các chủ xe cho thuê. Tại đó, các chủ xe có thể
	đăng tải thông tin về các xe mình cho thuê và khách thuê xe có thể lựa chọn các xe dựa trên loại xe, mẫu mã, hãng
	xe, thời gian thuê xe và có thể thanh toán cho chủ xe thông qua nhiều hình thức thanh toán khách nhau. Đây cũng là
	nhóm ứng dụng mà nhóm tác giả chọn để tham khảo, phân tích kỹ vì chúng rất sát so với định hướng của đề tài. Thách
	thức lớn nhất khi nghiên cứu nhóm ứng dụng này là các ứng dụng tại Việt Nam hầu như chỉ cho thuê ô tô, không có xe
	máy; còn các ứng dụng khác trên thế giới thì chưa có mặt hoạt động tại Việt Nam. Nhóm tác giải đã chọn ra 3 ứng
	dụng để phân tích gồm \emph{Mioto}, \emph{Xego} và \emph{Sigo}. Nhìn chung, các hệ thống này có phương thức nghiệp
	vụ tương tự nhau.

	\subsection{Ứng dụng thuê xe Mioto}

	Mioto là ứng dụng được phát triển bởi Công ty Cổ Phần Mioto Asia và cũng là sản phẩm chính của công ty này. Mioto
	Asia là công ty hoạt động trong lĩnh vực cho thuê xe tự lái 4 -7 chỗ, theo mô hình kinh tế chia sẻ
	\footnote{
%		Công ty Mioto Asia. (2022). \emph{Về Mioto.} Truy cập ngày 05/06/2023 tại \url{https://www.mioto.vn/aboutus}
		\cite{aboutMioto}
	}.
	Theo định nghĩa của từ điển Cambridge, cụm từ \emph{"kinh tế chia sẻ"} có nghĩa là \emph{"một hệ thống kinh tế
	dựa trên hoạt động chia sẻ tài sản và dịch vụ giữa con người với nhau, thường được tổ chức trên nền tảng internet
	"}. Trong trường hợp này, tài sản được chia sẻ chính là các xe cho thuê.
	Ứng dụng đã được đưa vào sử dụng từ năm 2018, cho tới nay đã hoạt động được khoảng 5 năm và có mặt ở nhiều thành
	phố lớn tại Việt Nam, đóng góp vào phát triển du lịch trong nước. Hệ thống đã được đánh giá cao trên cả App Store
	và Google Play Store, với 4.9 sao tổng hợp từ hơn 6000 lượt đánh giá trên mỗi nền tảng. Hiện tại, Mioto đang
	hoạt động rất tích cực, đẩy mạnh quảng cáo trên các mạng xã hội và kết hợp với nhiều dịch vụ lớn như
	Traveloka \footnote{
		Một nền tảng cung cấp các dịch vụ du lịch lớn tại khu vực Đông Nam Á, phát triển bởi công ty Traveloka của
		Indonesia.
	}
	hay Momo \footnote{
		Nền tảng ví điện tử tích hợp nhiều dịch vụ tiện ích, phát triển bởi công ty M\_Service của Việt Nam.
	}.

	Ưu điểm:
	\begin{itemize}
		\item Hệ thống hỗ trợ đa nền tảng, có thể truy cập thông qua trang web lẫn mobile app trên Android và iOS.
		\item Theo công cụ Wappalyzer, trang web của hệ thống được phát triển từ React.js, một bộ thư viện chuyên đề
		phát triển các ứng dụng web. Kết hợp với việc hệ thống có giao diện riêng cho ứng dụng di động, có thể suy ra
		được hệ thống sử dụng bộ tech-stack lớn, tách biệt giữa giao diện người dùng và logic hệ thống. Điều này cho
		thấy hệ thống Mioto được đầu tư kỹ lưỡng và nổi trội hơn so với các hệ thống khác, vốn chỉ sử dụng Wordpress
		để xây dựng trang chủ của hệ thống.
		\item Hệ thống có giao diện được thiết kế tốt và dễ sử dụng, đặc biệt là phần ứng dụng trên nền web.
		\item Hệ thống hiển thị chi tiết hơn thông tin về các xe cho thuê. Vì business của hệ thống tạo điều kiện dễ
		dàng hơn cho các chủ xe tư nhân thay vì chỉ tập trung vào các đại lý cho thuê xe nên người dùng có thể xem
		được kỹ hơn thông tin về từng chiếc xe.
		\item Hệ thống hỗ trợ thanh toán với nhiều hình thức khác nhau: Momo, VNPay, AlePay, ZaloPay, Visa.
		\item Ứng dụng được cập nhật liên tục để cải thiện chất lượng.
	\end{itemize}

	Hạn chế:
	\begin{itemize}
		\item Hạn chế đầu tiên của ứng dụng này so với phạm vi đề tài nghiên cứu là nó chỉ tập trung vào thị trường
		xe ô tô và yêu cầu người dùng phải xác minh bằng lái ô tô để có thể sử dụng đủ các tính năng. Do đó, việc
		trải nghiệm thử ứng dụng sẽ khó được đủ sâu.
		\item Trang home của hệ thống trên cả trang web lẫn mobile app đều được hiện thực như một trang dashboard và
		gắn rất nhiều thông tin, tuy nhiên trên nền web thì người dùng có thể tập trung ngay vào phần đặt xe vì đã có
		sẵn các form cũng như có sẵn một số mẫu xe để chọn; còn trên nền mobile các nhà phát triển lựa chọn tách flow
		đặt xe sang một trang riêng nên trang chủ chỉ đóng vai trò landing page cho cả app, người dùng sẽ cần làm
		thêm một thao tác để đặt xe.
		\item Hệ thống chỉ hỗ trợ tiếng Việt.
	\end{itemize}

	\subsection{Ứng dụng XeGo}

	XeGo là ứng dụng thuê xe ô tô tự lái được sáng lập bởi Tập đoàn Công Nghệ NextTech - một trong những tập đoàn
	công nghệ lớn nhất Việt Nam, hỗ trợ kết nối người cho thuê ô tô và khách hàng cần thuê ô tô. Từ đó giúp người
	thuê nhanh chóng tìm được một chiếc xe ưng ý với mức giá tốt nhất, đáp ứng mong muốn giữa 2 bên. Hiện tại XeGo
	đã có mặt hơn 22 tỉnh và thành phố trên cả nước. Hệ thống được đánh giá 3.1 sao với khoảng 70 lượt đánh giá trên
	cả App Store và Google Play Store.

	\subsubsection{\textbf{Phương thức hoạt động, nghiệp vụ của ứng dụng}}

	Người dùng gồm 2 đối tượng chính:
	\begin{itemize}
		\item Người thuê xe - người có nhu cầu thuê xe tự lái cho mục đích cá nhân
		\item Chủ xe - người có nhu cầu cho thuê xe, kiếm thêm thu nhập
	\end{itemize}

	Ứng dụng là nơi kết nối giữa chủ xe và người thuê xe.

	Để sử dụng dịch vụ, cả chủ xe và người thuê xe phải tạo tài khoản trên ứng dụng. Quy trình nghiệp vụ của chủ xe
	và người thuê xe như sau:

	\begin{itemize}
		\item Người thuê xe:
		\begin{itemize}
			\item Đăng ký tài khoản: Người thuê xe thực hiện đăng ký tài khoản, sau đó cung cấp các thông tin cá nh
			ân cũng như hình ảnh các giấy tờ hợp pháp nhằm phục vụ cho việc xác định danh tính để có thể thực hiện
			việc thuê xe sau này.
			\item Tìm kiếm xe phù hợp với nhu cầu cần thuê: Lựa chọn địa điểm, thời gian, xe phù hợp rồi tiến hành
			tạo yêu cầu thuê xe.
			\item Thanh toán cọc: Sau khi đã gửi yêu cầu thuê xe, và được chủ xe xác nhận, người thuê sẽ tiến hành
			vi ệc thanh toán cọc. Việc thanh toán cọc có hỗ trợ nhiều phương thức thanh toán online khác nhau.
			\item Liên hệ chủ xe và tiến hành nhận xe: Sau khi đã đặt cọc, người thuê xe tiến hành liên hệ chủ xe để
			sắp xếp thời gian nhận xe.
		\end{itemize}

		\item Chủ xe:
		\begin{itemize}
			\item Đăng ký tài khoản: Chủ xe tiến hành việc đăng ký tài khoản trên ứng dụng, sau đó cung cấp các
			thông tin cá nhân cũng như hình ảnh các giấy tờ hợp pháp nhằm phục vụ cho việc xác định danh tính để có
			thể thực hiện việc cho thuê xe.
			\item Tiến hành đăng thông tin xe cho thuê: Sau khi ứng dụng đã xác nhận duyệt các thông tin của chủ xe,
			chủ xe có thể tiến hành cho thuê xe bằng cách chọn tính năng cho thuê xe, sau đó điền và cung cấp các
			thông tin hình ảnh yêu cầu.
			\item Tiến hành cho thuê xe: Sau khi ứng dụng duyệt các thông tin xe do chủ xe cung cấp, cũng như thông
			tin giá thuê xe, chủ xe đã có thể tiến hành cho thuê xe trên ứng dụng.
		\end{itemize}
	\end{itemize}

	\subsubsection{Ưu điểm}

	\begin{itemize}
		\item Ứng dụng dễ dàng sử dụng và thân thiện với người dùng.
		\item Có hỗ trợ đầy đủ các tính năng cho người thuê xe lẫn chủ xe.
		\item Đa dạng các loại xe cho khách hàng lựa chọn.
		\item Dễ dàng lọc và tìm xe phù hợp cho người thuê.
		\item Có tính năng ví điện tử tích hợp trong ứng dụng, hỗ trợ thanh toán điện tử dễ dàng hơn.
		\item Website thông tin về ứng dụng được hiện thực tốt, chứa đủ các thông tin có ích cho người dùng.
		\item Trang chủ có tích hợp một số tiện ích như Tra cứu phạt nguội, Mua bảo hiểm xe\ldots
		\item Trang chủ có tích hợp một số tiện ích như Tra cứu phạt nguội, Mua bảo hiểm xe \ldots
	\end{itemize}

	\subsubsection{Hạn chế}

	\begin{itemize}
		\item Chỉ hỗ trợ trên nền tảng mobile, chưa hỗ trợ cho nền tảng web.
		\item Giống như ứng dụng Mioto, tại trang chủ của ứng dụng có một số thông tin quảng cáo cho ứng dụng không
		cần thiết. Các thông tin này có thể để trong phần onboarding khi người dùng mở dứng dụng lần đầu tiện sau khi
		tải về.
		\item Một số nút bấm trên ứng dụng chưa được hiện thực tính năng hoặc hiện thực chưa đủ tốt (ví dụ nút ``Một
		số câu hỏi thường gặp'' không dẫn thẳng tới phần tương ứng trên website mà dẫn về trang chủ), gây khó hiểu cho
		người dùng.
		\item Ứng dụng đã lâu không được cập nhật (lần cuối cập nhật trên cả nền tảng Android và iOS là trong năm
		2022).
		\item Hệ thống chỉ hỗ trợ tiếng Việt.
	\end{itemize}

	\subsection{Ứng dụng Sigo}

	Sigo là ứng dụng cho thuê xe trung gian của Công ty TNHH SIGO Việt Nam được phát hành năm 2018. Ứng dụng là nơi
	kết nối chủ xe và khách thuê xe với nhau trên toàn quốc. Sigo là nền tảng cho thuê xe ứng dụng công nghệ, khác
	biệt hoàn toàn với cách thức thuê xe truyền thống. Với Sigo người dùng có thể tìm điểm cho thuê xe tự lái gần nhất
	một cách đơn giản ngay trên điện thoại hoặc máy tính cá nhân.Các chủ xe thông qua các xe của mình có thể cho thuê
	ngắn hạn kiếm tiền một cách dễ dàng. Còn khách hàng sẽ được trải nghiệm xe thoải mái và thuận tiện nhất. Hiện
	tại SIGO đã có mặt hầu hết các thành phố lớn trên cả nước. Tại thời điểm viết báo cáo này, ứng dụng chỉ vừa được
	release trên App Store và Google Play Store nên chưa có dữ liệu đánh giá.

	\subsubsection{Ưu điểm}
	\begin{itemize}
		\item Ứng dụng dễ dàng sử dụng và thân thiện với người dùng.
		\item Có hỗ trợ đầy đủ các tính năng cho người thuê xe lẫn chủ xe.
		\item Đa dạng các loại xe cho khách hàng lựa chọn.
		\item Dịch vụ thuê xe có hầu hết các thành phố lớn.
		\item Dễ dàng lọc và tìm xe phù hợp cho người thuê.
		\item Thông tin về giá thuê xe được minh bạch.
		\item Thông tin về chủ xe cũng như khách thuê xe cũng được minh bạch.
		\item Nền tảng web có giao diện tốt, đáp ứng được nhiều kích thước màn hình.
	\end{itemize}

	\subsubsection{Hạn chế}

	\begin{itemize}
		\item Ứng dụng trên nền tảng mobile đã lâu chưa cập nhật, nên không còn phù hợp với các phiên bản mới của hệ
		điều hành Andriod và iOS. Phần cài đặt ứng dụng trên web cũng không dẫn được tới trang ứng dụng này trên Play
		Store và App Store.
		\item Nhiều nút bấm hoặc đường dẫn trên trang web không dẫn người dùng tới các trang thông tin cần thiết mà dẫn
		về trang chủ, gây khó chịu cho người dùng khi muốn tìm kiếm các thông tin đó.
		\item Hệ thống chỉ hỗ trợ tiếng Việt.
		\item Phiên bản Android của ứng dụng chỉ vừa mới được release gần đây (24/9/2023) và chưa sử dụng được.
	\end{itemize}


% \begin{center}
% \begin{table}[!ht]
%     \centering
%     \begin{tabular}{| c | c | c | c |}
%         \hline
%         STT & Ứng dụng & Nền  tảng & Ưu điểm \\
%         \hline
%         1 & Mioto & Web, Android, iOS & \\
%         \hline
%         2 & Xego & Android, iOS & \\
%         \hline
%         3 & Sigo & Web & \\
%         \hline
%     \end{tabular}
%     \caption{So sánh giữa các ứng dụng thuê xe trung gian}
%     \label{tab:my_label}
% \end{table}
% \end{center}

% \newpage


	\section{Tính cấp thiết của đề tài - Lý do chọn đề tài}

	Thông qua việc tìm hiểu thực tiễn một số ứng dụng hiện có trên thị trường như trên, có thể thấy nhu cầu thuê xe tại
	Việt Nam có tiềm năng, nhưng chưa thực sự phát triển. Các doanh nghiệp trong lĩnh vực này chọn hướng tới an toàn
	hơn là thu6 xe ô tô mà vẫn còn bỏ ngỏ mảng xe máy. Do đó, việc phát triển một hệ thống thuê xe máy tự lái trung
	gian là cần thiết. Bên cạnh đó, các ứng dụng được chọn ra để tham khảo vẫn còn nhuững điểm hạn chế cần khắc phục;
	cũng như các tính năng, giao diện của chúng có phần tương đồng, học hỏi lẫn nhau khá nhiều.

	Chi tiết về các tính năng của hệ thống sẽ được mô tả trong chương \ref{chap:requirements}.


	\section{Phạm vi của đề tài}

	Đề tài chỉ tập trung vào quá trình phát triển hệ thống cho thuê xe máy trung gian trong một thành phố, do đó các
	vấn đề mà đề tài sẽ trình bày bao gồm việc tìm hiểu và phân tích các yêu cầu của hệ thống, thiết kế hệ thống, phát
	triển và kiểm thử hệ thống. Các vấn đề khác như pháp lý, kinh doanh không nằm trong phạm vi của đề tài.

	Để phù hợp với nguồn lực của nhóm, đề tài hiện tại hướng tới việc cho thuê xe máy trong một thành phố. Thành phố Đà
	Lạt được chọn làm thành phố thí điểm để chuẩn bị một số dữ liệu cho đề tài vì đây là thành phố mang nhiều đặc điểm
	phù hợp để có thể triển khai hệ thống. Đà Lạt vừa là thành phố du lịch nổi tiếng, vừa có địa hình đặc biệt với
	nhiều dốc và đường hẻm, nên nhu cầu cho thuê xe máy tại đây cũng rất cao. Do đó, đây là địa điểm phù hợp để thí
	điểm cho đề tài.


	\section{Kết cấu đề tài}

	Đề tài bao gồm 5 chương:

	\begin{itemize}
		\item \textbf{Chương \ref{chap:intro}: Giới thiệu đề tài} - giới thiệu về nhu cầu thực tế, các ứng dụng tương tự
		trên thị trường, ý nghĩa thực tiễn và phạm vi của đề tài.
		\item \textbf{Chương \ref{chap:requirements}: Phân tích yêu cầu} - Phân tích sâu hơn về các yêu cầu chức năng và phi
		chức năng của hệ thống, mô hình hóa các yêu cầu chức năng bằng lược đồ use-case.
		\item \textbf{Chương \ref{chap:system_design}: Thiết kế hệ thống} - Trình bày các khía cạnh khác nhau của hệ thống,
		bao gồm kiến trúc, hành vi, tương tác giữa các thành phần, giao diện, dữ liệu.
		\item \textbf{Chương \ref{chap:implementation}: Kế hoạch hiện thực} - Trình bày về các cơ sở lý thuyết để hiện thực
		ứng dụng, chủ yếu sẽ là các công nghệ dự kiến được sử dụng.
		\item \textbf{Chương \ref{chap:conclusion}: Tổng kết} - Tổng kết lại các vấn đề đã trình bày và đánh giá thành quả,
		đồng thời nêu thêm một số hướng phát triển của hệ thống.
		% \item \textbf{Chương 6: Danh mục tài liệu tham khảoj}.

	\end{itemize}

	\newpage

\end{document}
