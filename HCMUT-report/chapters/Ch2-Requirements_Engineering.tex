%! Compiler = lualatex --shell-escape
%! BibTeX Compiler = biber

\documentclass[../main.tex]{subfiles}
\graphicspath{{\subfix{../figures/}}}

\begin{document}

	\chapter{Phân tích yêu cầu}
	\label{chap:requirements}

	\justifying


	\section{Các chức năng dự kiến của ứng dụng}

	Đi từ những nhu cầu trên kết hợp với một số đặc thù của việc thuê xe máy, nhóm tác giả đã đưa ra các nhóm người dùng
	và các chức năng mà hệ thống cần có để phục vụ các nhóm người dùng đó như sau:

	\subsection{Các tính năng dành cho khách thuê xe}

	Đây là nhóm khách hàng có nhu cầu tìm kiếm và thuê xe tự lái trong một khoảng thời gian nhất định. Khách thuê xe có
	thể sử dụng các tính năng sau:

	\begin{itemize}
		\item Khách có nhu cầu thuê xe có thể tìm thông tin các xe đang được cho thuê theo các tiêu chí minh đặt ra (mẫu xe,
		thời gian có thể cho thuê, giá cho thuê, đánh giá xe và chủ xe\ldots) để tìm được phương tiện phù hợp nhất cho kế
		hoạch sử dụng cá nhân
		\item Khách thuê xe sau khi lựa chọn được (những) xe phù hợp để thuê có thể đặt xe theo lịch trình đã chọn sẵn (thời
		gian, địa điểm nhận và trả xe, giấy tờ\ldots)
		\item Khách thuê xe có thể chọn thêm một số tiện ích kèm theo từng xe cho thuê nếu có (mũ bảo hiểm, phụ tùng, bảo
		hiểm xe \ldots)
		% TODO: consider this
		\item Khách thuê xe có thể thuê được nhiều xe để phù hợp với kế hoạch cá nhân (ví dụ đi du lịch nhóm, chỉ cần 1
		người đại diện thuê nhiều xe).
		\item Khách thuê xe trước khi tiến hành đặt xe cần phải đăng ký/đăng nhập và cung cấp các thông tin cá nhân (giấy tờ
		liên quan đến điều khiển xe) cần thiết để chứng minh mình có khả năng sử dụng xe được thuê một cách hợp pháp.
		\item Khách thuê xe sau khi đẵ đăng nhập và cung cấp đủ các thông tin có thể chọn phương thức thanh toán phù hợp để
		xác nhận thanh toán.
		\item Khách thuê xe có thể nhận thông báo về quyết định cho thuê hoặc từ chối của chủ xe.
		\item Khách thuê xe sau khi thanh toán có thể nhận thông báo khi gần tới thời điểm nhận và trả xe.
		\item Khách thuê xe có thể nhắn tin với chủ xe để được hỗ trợ khi cần.
		\item Khách thuê xe có thể đưa ra đánh giá hoặc báo cáo về chủ xe và xe cho thuê.
		\item Khách thuê xe trong thời gian đặt thuê xe tới trước khi thời gian thuê xe kết thúc (bao gồm trước khi nhận xe,
		tại thời điểm nhận xe, trong thời điểm đang sử dụng xe) có thể huỷ thuê xe vì những lý do khác nhau.
		\item Khách thuê xe có thể xem lại lịch sử các chuyến thuê xe của mình.
		\item Khách thuê xe có thể đăng ký thuê xe khi có nhu cầu (chuyển sang vai trò chủ xe cho thuê)
		\item Khách thuê xe có thể xoá tài khoản và các thông tin của mình khi không còn nhu cầu sử dụng ứng dụng.
	\end{itemize}

	\subsection{Các tính năng dành cho chủ xe cho thuê}

	Đây là nhóm người dùng có nhu cầu cho thuê xe của mình nhằm kiếm thêm thu nhập. Chủ xe cho thuê có thể sử dụng các
	tính năng sau:

	\begin{itemize}
		\item Chủ xe có thể đăng ký cho thuê xe trên ứng dụng bằng cách cung cấp các thông tin mô tả và thông tin pháp lý
		của xe để có thể kiếm thêm thu nhập.
		\item Chủ xe có thể cung cấp thêm các tiện ích đi kèm xe mà mình cho thuê.
		\item Chủ xe có thể cập nhật các thông tin về xe (tình trạng cho thuê hay không - trong lúc không có khách thuê) vì
		những lý do khách nhai.
		\item Chủ xe có thể nhận được thông báo khi có khách gửi yêu cầu thuê xe của mình.
		\item Chủ xe có thể xem một số thông tin cần thiết về yêu cầu thuê xe (đánh giá về người khách thuê xe, thời gian,
		địa điểm giao, nhận xe, phương thức thanh toán\ldots) và chấp nhận hoặc từ chối yêu cầu thuê xe này.
		\item Chủ xe cần được thông báo khi gần tới thời điểm nhận và trả xe.
		\item Chủ xe có thể xác nhận xe đã được giao và được trả về.
		\item Chủ xe có thể đưa ra đánh giá hoặc báo cáo về người khách vừa thuê xe của mình.
		\item Chủ xe cũng có thể chuyển sang vai trò người thuê xe để thuê xe khi có nhu cầu.
		\item Chủ xe có thể chat với người thuê xe hoặc người có nhu cầu thuê xe để tư vấn và hỗ trợ cho họ khi họ cần.
		\item Chủ xe có thể xam lại lịch sử cho thuê xe của mình.
		\item Chủ xe có thể xoá tài khoản khi không còn nhu cầu sử dụng.
	\end{itemize}

	\subsection{Các tính năng dành cho Officer - đội ngũ vận hành hệ thống}

	Đây là nhóm người dùng đóng vai trò vận hành hệ thống và thực hiện các tác vụ mà các khách hàng không có thẩm quyền.

	\begin{itemize}
		\item Officer có thể xem các dashboard của hệ thống.
		\item Officer có thể xem danh sách người dùng và thông tin chi tiết người dùng.
		\item Officer có thể xem danh sách các xe đăng ký cho thuê và thông tin chi tiết.
		\item Officer có thể xem danh sách các cuộc đặt xe và thông tin chi tiết.
		\item OFficer có thể duyệt hoặc từ chối duyệt thông tin liên quan tới bằng lái và đăng ký xe.
		\item Officer có thể xoá tài khoản người dùng khi có yêu cầu.
	\end{itemize}


	\section{Các yêu cầu phu nghiệp vụ của ứng dụng}

	Các yêu cầu phi nghiệp vụ (hay Non-functional requirements - NFR) là các yêu cầu không liên quan trực tiếp tới logic
	nghiệp vụ của ứng dụng, nhưng cần thiết để ứng dụng trở nên đáng tin cậy và giúp người dùng có trải nghiệm sử dụng tốt
	hơn. Với hệ thống này, nhóm tác giả đưa ra các yêu cầu phi nghiệp vụ được chia thành các nhóm như sau:

	\subsection{Usability - Tính dễ sử dụng}
	\begin{itemize}
		\item Hệ thống cần phải được thiết kế nhất quán về màu sắc, phông chữ, kích cỡ\ldots
		\item Hệ thống cần đảm bảo có các trang thông tin hỗ trợ cho người dùng.
		\item  Hệ thống cần được thiết kế sao cho người dùng có thể dễ dàng thực hiện được các luồng hoạt động chính, chỉ
		tốn khoảng 30 giây cho việc hiểu các hướng dẫn trên màn hình hoặc các trường dữ liệu trên form.
	\end{itemize}

	\subsection{Performance - Hiệu năng}
	\begin{itemize}
		\item Với mỗi tương tác của người dùng, hệ thống cần phản hồi trong vòng 2 giây, tối đa là 3 giây cho một số
		thao tác như đăng nhập bằng Google
	\end{itemize}

	\subsection{Reliability - Tính đáng tin cậy}
	\begin{itemize}
		\item Hệ thống cần phải hiển thị đúng các thông tin người dùng cung cấp
		\item Hệ thống cập nhật đúng và đủ những thông tin người dùng cung cấp
	\end{itemize}

	\subsection{Security - Tính bảo mật}
	\begin{itemize}
		\item Người dùng chỉ xme được các thông tin được cho phép, khogn6 thể xme các thông tin cá nhân của người dùng
		khác hoặc của role khác.
		\item Khi người dùng yêu cầu một số thao tác quan trọng như thay đổi giấy tờ, đổi mật khẩu\ldots người dùng cần
		phải xác nhận mật khẩu.
		\item Các tầng của ứng dụng đều cần phải validate dữ liệu đầu vào.
	\end{itemize}

	\subsection{Compatibility - Khả năng tương thích}
	\begin{itemize}
		\item Hệ thống cần đáp ứng nhiều kích cỡ màn hình mà vẫn đảm bảo được bố cục nội dung được hiển thị hợp lý. Các kích
		thước mà hình tiêu chuẩn là màn hình điện thoại thông minh, màn hình máy tính cá nhân và màn hình máy tính bảng.
		\item Hệ thống trước mắt cần phải sử dụng được trên nền tảng web để có thể sử dụng được trên cả máy tính cá nhân lẫn
		điện thoại thông minh.
		\item Hệ thống phải tương thich 1duoc975 với các trình duyệt web lớn: \emph{Google Chrome, Microsoft Edge, Safari,
			Firefox}.
	\end{itemize}

	\subsection{Localization - Tính bản địa hoá}
	\begin{itemize}
		\item Để phục vụ mục đích du lịch, trước may81 hệ thống cần hỗ trợ 2 ngôn ngữ là tiếng Việt và tiếng Anh. Các ngôn
		ngữ khác có thể được thêm vào sau này.
	\end{itemize}


	\section{Thiết kế lược đồ use-case}

% \section{Mô tả các use-case}

	Sau đây là tổng quan lược đồ use-case của hệ thống

	\begin{figure}[ht]
		\centering
		\includegraphics[scale=0.6]{images/usecase/usecase-1.png}
		\caption{Lược đồ use-case cho người dùng khi chưa đăng nhập}
		\label{fig:usecase_1}
	\end{figure}

	Cho người dùng khi chưa đăng nhập, ta có các use-case:

	\begin{itemize}
		\item Đăng nhập
		\item Đăng ký
		\item Tìm kiếm xe
		\item Xem thông tin xe cho thuê
		\item Lấy lại mật khẩu
		\item Xác thực thông tin đăng ký
	\end{itemize}

	\begin{figure}[ht]
		\centering
		\includegraphics[scale=0.4, angle=270, origin=c]{images/usecase/usecase-21.png}
		\caption{Lược đồ use-case cho người dùng khi đã đăng nhập (1)}
		\label{fig:usecase_21}
	\end{figure}

	\begin{figure}[ht]
		\centering
		\includegraphics[scale=0.5]{images/usecase/usecase-22.png}
		\caption{Lược đồ use-case cho người dùng khi đã đăng nhập (2)}
		\label{fig:usecase_22}
	\end{figure}

	Cho người dùng khi đã đăng nhập, ta có các use-case:

	\begin{itemize}
		\item Tìm kiếm xe
		\item Xem thông tin xe cho thuê
		\item Thuê xe
		\item Xác thực thông tin khách thuê xe
		\item Xác nhận các thông tin
		\item Điền thông tin cho thuê xe
		\item Chọn các phụ kiện, tiện ích
		\item Chọn phương thức thanh toán
		\item Xem thông báo
		\item Xác nhận đã giao xe từ người cho thuê
		\item Xác nhận đã nhận xe từ người cho thuê
		\item Xác nhận/từ chối cho thuê xe
		\item Xem danh sách xe đã thuê
		\item Đánh giá chủ xe, xe cho thuê
		\item Xem các xe đang thuê
		\item Huỷ thuê xe
		\item Xem danh sách tin nhắn
		\item Nhắn tin với người dùng
		\item Chỉnh sửa thông tin cá nhân
		\item Xác thực mật khẩu
		\item Xoá tài khoản
		\item Đăng ký cho thuê xe
		\item Cung cấp thông tin xe
		\item Cung cấp thông tin các tiện ích đi kèm
		\item Xem danh sách xe đang cho thuê
		\item Cập nhật thông tin xe cho thuê
		\item Xem lịch sử cho thuê xe
		\item Đánh giá người thuê xe
		\item Xem thông tin người dùng khác
		\item Đăng xuất
	\end{itemize}

	\begin{figure}[ht]
		\centering
		\includegraphics[scale=0.6, angle=270, origin=c]{images/usecase/usecase-3.png}
		\caption{Lược đồ use-case cho officer}
		\label{fig:usecase_3}
	\end{figure}

	Officer có các use-case sau:

	\begin{itemize}
		\item Xem danh sách thông tin người dùng cung cấp
		\item Duyệt thông tin người dùng cung cấp
		\item Từ chối duyệt thông tin người dùng cung cấp
		\item Cung cấp lý do từ chối
		\item Xem các dashboard hệ thống
		\item Xem danh sách người dùng trên hệ thống
		\item Tìm kiếm người dùng
		\item Gửi thông báo cho người dùng
		\item Xoá tài khoản người dùng
		\item Lưu thông tin cá nhân, không cho sử dụng lại dịch vụ
		\item Đăng xuất
	\end{itemize}

	\newpage

\end{document}
