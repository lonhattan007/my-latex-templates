%! Compiler = lualatex --shell-escape
%! BibTeX Compiler = biber

\documentclass[../main.tex]{subfiles}
\graphicspath{{\subfix{../figures/}}}

\begin{document}

	\justifying


	\chapter{Kế hoạch hiện thực}
	\label{chap:implementation}


	\section{Công nghệ hiện thực}

	\subsection{Hướng tiếp cận đa nền tảng}

	Đối với một ứng dụng tương tác với người dùng hiện nay, có nhiều nền tảng có thể được lựa chọn tùy vào nhu cầu ứng
	dụng đó giải quyết và nhóm đối tượng người dùng mà ứng dụng hướng tới. Trên các thiết bị hiện nay, các nền tảng lớn để
	phát triển ứng dụng bao gồm Web app, Desktop app, và Mobile app. Dựa trên nhu cau62 phục vụ nhóm khách hàng mà ứng
	dụng hướng tới, ứng dụng được ưu tiên chạy được trên nền tảng web và mobile. Để chọn được hướng tiếp cận đa nền tảng
	hiệu qua, ta sẽ so sánh một số khía cạnh của các nền tảng trên.

	Web app là các ứng dụng chạy trên nền web. Để sử dụng các ứng dụng này, người dùng cần phải truy cập và tương tác
	thông qua một trình duyệt web (web browser). Điểm thuận lợi nhất của các ứng dụng trên nền tảng này là tính tương
	thích rộng. Hầu hết các thiết bị ngày nay đều có hỗ trợ trình duyệt web, từ máy tính cá nhân, máy tính bảng, điện
	thoại di động cho tới một số thiết bị nhúng như trên một số phương tiện hiện đại (xe ô tô, TV). Người dùng chỉ cần
	truy cập đến địa chỉ trang web mà không cần phải trải qua quá trình cài đặt. Trên các phiên bản mới của các trình
	duyệt sử dụng nhâm Chromium hiện nay còn có hỗ trợ công nghệ PWA (Progressive Web App). Một ứng dụng PWA có thể được
	cài đặt trên các thiết bị như một ứng dụng riêng lẻ trên điện thoại thông minh hoặc máy tính cá nhân. Ứng dụng này có
	một số tính năng hoạt động được khi không có mạng cũng như có thể tương tác với phần cứng của thiết bị. Với hướng tiếp
	cận này, chỉ cần xây dựng một code-base cho phần giao diện của ứng dụng sao cho đảm bảo được tính responsive hoặc tính
	adaptive thì có thể đảm bảo ứng dụng có thể được sử dụng trên nhiều thiết bị khác nhau, giúp tiết kiệm chi phí phát
	triển ứng dụng. Hướng tiếp cận này phù hợp với các ứng dụng nhỏ, cần tương tác nhiều với internet, hoặc cho giai đoạn
	đầu trong quá trình phát triển ứng dụng, khi mà cần đưa ra một ứng dụng MVP có độ phủ rộng được xem là ưu tiên. Đây
	cũng là hướng tiếp cận nhóm nhận định là phù hợp với hệ thống cần xây dựng.

	\subsection{Lựa chọn công nghệ xây dụng \Gls{frontend}}

	\subsubsection{Typescript - Ngôn ngữ lập trình}

	Khi lập trình phần \Gls{frontend} của các ứng dụng web hiện đại, Javascript là ngôn ngữ lập trình chính được sử dụng
	cùng với ngôn ngữ đánh dấu HTML và ngôn ngữ định kiểu CSS. HTML tạo ra các thành phần, hay có thể gọi là ``khung
	xương'' của trang web, CSS giúp trang trí các thành phần đó để trang web hiển thị đẹp hơn, và Javascript điều khiển
	các logic bên trong trang web, giúp cho người dùng có thể tương tác với hệ thống. Từ phiên bản ES6 trở đi, Javascript
	đã hỗ trợ lập trình hướng đối tượng, tuy nhiên đây vẫn là một ngôn ngữ không có ràng buộc kiểu tĩnh nên nó có thể phát
	sinh nhiều lỗi trong thời gian runtime của ứng dụng, cũng như làm cho việc kiểm thử mã nguồn phức tạp hơn. Để khắc
	phục vấn đề đó, công ty Microsoft đã tạo ra một superset của Javascript, đặt tên là Typescript.

	\begin{figure}[ht]
		\centering
		\includegraphics[scale=0.3]{images/techstacks/typescript_logo}
		\caption{Logo của Typescript}
		\label{fig:ts-logo}
	\end{figure}

	Vì là superset của Javascript nên cú pháp của Typescript phần lớn giống với Javascript. Điểm khác biệt chính nằm ở
	việc Typescript là ngôn ngữ có ràng buộc kiểu tĩnh. Nhà phát triển cần phải khai báo rõ kiểu của các biến, hàm và từ
	đó có thể dễ dàng phát hiện lỗi hơn ngay trong thời gian viết mã nguồn. Typescript cũng cung cấp một số tính năng khác
	như enums, generic, interface để hỗ trợ lập trình hướng đối tượng, optional type để dễ tuỳ biến trong khai báo\ldots
	Các cải tiến này đã khiến cho Typescript trở thành ngôn ngữ được sử dụng trong nhiều dự án lớn để tăng tính
	scalability và maintainability. Với việc dần được sử dụng rộng rãi để thay thế cho Javascript, rất nhiều thư viện
	Javascript đều đã hỗ trợ Typescript, đặc biệt là các thư viện lớn. Các công nghệ nhóm chọn sử dụng bên dưới cũng không
	phải là ngoại lệ.

	Từ các ưu điểm kể trên, nhóm đã quyết định sử dụng Typescript để phát triển hệ thống.

	\subsubsection{React.js - Thư viện xây dựng giao diện web}

	React.js là một những thư viện hỗ trợ xây dựng giao diện web được sử dụng phổ biến nhất hiện nay. Đây là một bộ thư
	viện được phát triển bởi Meta (công ty mẹ của Facebook) và được cộng đồng đóng góp từ năm 2013, cho đến nay đã ra mắt
	tới phiên bản 18.0. React.js hỗ trợ phát triển theo hướng component.

	\begin{figure}[ht]
		\centering
		\includegraphics[scale=0.3]{images/techstacks/react_logo}
		\caption{Logo của React.js}
		\label{fig:react-logo}
	\end{figure}

	Nhóm chọn sử dụng React.js vì những lý do sau:

	\begin{itemize}
		\item React.js cung cấp nhiều tính năng hữu ích so với việc viết template HTML truyền thống, đặc biệt là tư duy
		component-based giúp cho việc tái sử dụng mã nguồn trở nên dễ dàng hơn.
		\item Đây là bộ thư viện phổ biến, được phát triển bởi tập đoàn lớn cũng như có cộng đồng rộng rãi. Vì vậy, bộ tài
		liệu hướng dẫn của thư viện này được viết rất đầy đủ, chỉn chu. Cộng đồng hỗ trợ giải đáp những khúc mắc khi sử
		dụng React.js cũng rất lớn nên ta có thể dễ dàng tìm kiếm thông tin cần thiết.
		\item Thành viên nhóm đã có kinh nghiệm sử dụng React.js trước đó, vì vậy nên thời gian chuẩn bị cho giai đoạn hiện
		thực cũng được giảm đi đáng kể.
	\end{itemize}

	Tuy nhiên, bản thân React.js chỉ là thư viện xây dựng giao diện. Thư viện này không hỗ trợ các tác vụ như routing,
	quản lý trạng thái\ldots Để làm được những điều đó, ta cần phải sử dụng thêm một số thư viện khác sẽ được trình bày ở
	phần sau. React.js cũng không có ràng buộc về xây dựng cấu trúc dự án. Về một mặt, nó có thể giúp cho nhà phát triển
	tuỳ ý cấu trúc dự án; nhưng mặt khác, việc tự xây dựng một cấu trúc tối ưu, dễ mở rộng, đảm bảo mã nguồn sạch (clean
	code) lại là tương đối phúc tạp đối với các nhà phát triển chưa có kinh nghiệm. Vì thế, nhóm sẽ không chỉ sử dụng
	React.js thuần tuý, mà sẽ sử dụng một ``meta-\Gls{framework}'' dựa trên nó là Next.js.

	\subsubsection{Next.js - \Gls{framework} xây dựng giao diện web}

	Next.js là một \Gls{framework} (khung thức) xây dựng giao diện web, được phát triển bởi công ty Vercel và được cộng
	đồng mã nguồn mở đóng góp từ năm 2016. Cho tới nay, Next.js đã ra mắt phiên bản 13 với nhiều thay đổi đáng chú ý so
	với những phiên bản trước đó.

	\begin{figure}[ht]
		\centering
		\includegraphics[scale=0.3]{images/techstacks/next_logo}
		\caption{Logo của Next.js}
		\label{fig:next-logo}
	\end{figure}

	Như đã nói trên, Next.js là một \Gls{framework} xây dựng giao diện web được phát triển dựa trên React.js. Về cơ bản,
	các nhà phát triển có thể xây dựng các thành phần web trong Next.js giống như khi sử dụng React.js và cú pháp JSX.
	Next.js cung cấp cho người dùng một số component được xây dựng sẵn, thay thế và được tối ưu hơn một số web element có
	sẵn. Tuy nhiên, điểm đáng chú ý nhất của Next.js chính là cơ chế file-based routing (định tuyến dựa trên tên file) của
	nó. Thay vì nhà phất triển phải cấu hình thủ công cho các route hay URI trong ứng dụng của mình tới các trang, bộ biên
	dịch (compiler) của Next.js hỗ trợ tạo các URI theo đường dẫn tới các file trong dự án. Trong Next.js 13, các file nằm
	trong tập tin \emph{app} đều được quản lý bởi cơ chế này, với \emph{app} tương ứng với đường dẫn \emph{root} hay
	\emph{domain/} (domain tương ứng với tên miền khi triển khai của ứng dụng). Ví dụ như file
	\emph{app/dashboard/users/index.tsx} sẽ được triển khai tại đường dẫn \emph{domain/dashboard/users}

	Một tính năng đáng chú ý nữa của Next.js là nó hỗ trợ Server-side Rendering và Server-side Generation hay có thể hiểu
	là tạo các trang tĩnh. React.js thuần thường sử dụng cơ chế Single Page Application hay ứng dụng đơn trang. Nó bao gồm
	việc gửi về trình duyệt người dùng một file HTML có gắn kèm một file Javascript. File Javascript này sau đó sẽ tạo ra
	các thành phần của trang web và điều khiển logic của ứng dụng. Phương pháp này có thể dẫn tới giảm sút về hiệu năng
	nếu file Javascript quá lớn. Next.js hỗ trợ nhà phát triển tạo ra một ``mẫu'' (template) cho một lượng lớn các trang
	bằng cách định nghĩa nội dung và dữ liệu mà các trang đó cần. Bộ phiên dịch Next.js sẽ dựa vào lượng thông tin thu
	được để tạo sẵn các trang tĩnh trên phía \Gls{frontend} server. Khi người dùng truy cập tới các trang này, các file
	HTML tĩnh sẽ được gửi về trình duyệt với tốc độ nhanh hơn. Kỹ thuật này cũng giúp tăng cường SEO (Search Engine
	Optimization - Tối ưu hoá công cụ tìm kiếm) của hệ thống. Nhà phát triển có thể quyết định những route nào sẽ sử dụng
	cơ chế render nào.

	Nhóm chọn sử dụng Next.js vì những lý do sau:

	\begin{itemize}
		\item Đây là công cụ được sử dụng khá phổ biến và được đánh giá cao trong những năm gần đây, đặc biệt là nó cũng
		được đề xuất trên chính trang tài liệu của React.js \footnote{Start a new React Project. Truy cập ngày 01/12/2023
		tại \url{https://react.dev/learn/start-a-new-react-project}}.
		\item Next.js có bộ tài liệu đầy đủ và cộng đồng hỗ trợ lớn mạnh, giúp cho nhóm dễ dàng giải đáp khúc mắc.
		\item Cơ chế file-based routing của Next.js rất dễ đọc, dễ hiểu, từ đó giảm tải bớt một phần tác vụ cấu hình routing
		thường gặp khi phát triển web app.
		\item Next.js hỗ trợ nhiều cơ chế render như SSR, SSG, client-side, giúp cho nhà phát triển có thể linh hoạt sử dụng
		cho các tình huống khác nhau.
		\item Cơ chế file-based routing nói trên cũng đã định hình được cấu trúc của dự án, giúp giảm tải cho nhà phát
		triển.
	\end{itemize}

	\subsubsection{Redux - Thư viện quản lý trạng thái}

	Redux là một bộ thư viện mã nguồn mở hỗ trợ quản lý trạng thái (state) cho các ứng dụng web. Redux được phát triển vào
	năm 2015 bởi Dan Abramov và Andrew Clark, hai nhà phát triển thuộc nhóm cốt lõi phát triển React.js tại Meta, và sau
	đó được cộng đồng mã nguồn mở đóng góp. Đây là một trong những thư viện quản lý trạng thái phổ biến nhất.

	\begin{figure}[ht]
		\centering
		\includegraphics[scale=0.3]{images/techstacks/redux_logo}
		\caption{Logo của Redux}
		\label{fig:redux-logo}
	\end{figure}

	State (trạng thái) là một khái niệm cơ bản trong React.js. State là những thuộc tính của các component mà khi những
	thuộc tính này thay đổi giá trị, các component cũng sẽ được render lại. Trong thực tế sử dụng, các trạng thái này
	thường ánh xạ tới dữ liệu thật và được hiển thị trên trang web để người dùng có thể thấy và tương tác. React.js cung
	cấp các hook useState, useContext và cho phép người dùng tạo ra các custom hook để hỗ trợ quản lý trạng thái, đảm bảo
	cho các trạng thái được đòng bộ giữa các component. Tuy nhiên, trong các ứng dụng lớn, việc quản lý trạng thái thủ
	công như vậy khá phức tạp và dễ sinh ra lỗi. Redux giải quyết vấn đề này bằng cách tạo ra các store như một database
	của riêng phía \Gls{frontend} để lưu trữ giá trị của các state, và các phương thức để các component có thể truy xuất
	và thay đổi giá trị của các state đó.

	\begin{figure}[ht]
		\centering
		\includegraphics[scale=0.3]{images/techstacks/redux_arch}
		\caption{Kiến trúc Redux}
		\label{fig:redux-arch}
	\end{figure}

	Hình \ref{fig:redux-arch} biểu diễn chiều dữ liệu trong một ứng dụng có sử dụng Redux. UI chính là các component
	được viết từ React.js và sử dụng state được lưu trong Redux store. Khi người dùng tương tác với các component, ví dụ
	như click chọn các lựa chọn, các sự kiện này sẽ ``gửi đi'' (dispatch) một đối tượng tới Redux store. Dựa vào các thuộc
	tính trong đối tượng này, store sẽ biết các state nào cần được thay đổi và thay đổi như thế nào để cập nhật đúng các
	state đó. Khi các state này thay đổi, component cũng sẽ được render lại để hiển thị đúng thông tin.

	Nhóm quyết định sử dụng Redux vì những lý do sau:

	\begin{itemize}
		\item Redux cung cấp giải pháp quản lý trạng thái dễ mở rộng, dễ bảo trì về sau cho dự án.
		\item Redux cung cấp sẵn kiến trúc quản lý trạng thái dễ đọc, dễ hiểu, giúp cho nhà phát triển dễ dàng cấu trúc dự
		án.
		\item Redux cung cấp các công cụ phát triển ở nhiều nền tảng như trên trình duyệt để kiểm thử trong thời gian
		runtime, trên IDE để hổ trợ viết mã nguồn nhanh chóng.
		\item Redux có hệ thống tài liệu đẩy đủ và cộng đồng hổ trợ lớn mạnh.
	\end{itemize}

	\subsubsection{AntDesign - Thư viện thành phần giao diện web}

	AntDesign là một hệ thống thiết kế (design system) cho các ứng dụng web, cung cấp sẵn một lượng lớn các component phổ
	biến cũng như hỗ trợ tuỳ biến các component đó. AntDesign được phát triển bởi Ant Group (công ty mẹ của Alibaba) từ
	năm 2016.

	\begin{figure}[ht]
		\centering
		\includegraphics[scale=0.3]{images/techstacks/antd_logo}
		\caption{Logo của AntDesign}
		\label{fig:antd-logo}
	\end{figure}

	\subsection{Lựa chọn công nghệ xây dựng \Gls{backend}}

	\subsubsection{Nest.js - \Gls{framework} xây dựng API}

	Nest.js là một \Gls{framework} phát triển phần \Gls{backend} phổ biến, được xây dựng dựa trên ngôn ngữ Typescript và
	lấy cảm hứng từ hai framework nổi tiếng là Express.js và Angular.js. Nest.js được tạo ra bởi nhà phát triển người Ba
	Lan Kamil Mysliwiec vào năm 2017.

	\begin{figure}[ht]
		\centering
		\includegraphics[scale=0.3]{images/techstacks/nest_logo}
		\caption{Logo của Nest.js}
		\label{fig:nest-logo}
	\end{figure}

	Nhóm quyết định sử dụng Nest.js bởi những lý do sau:

	\begin{itemize}
		\item Nest.js là \Gls{framework} sử dụng Typescript là ngôn ngữ lập trình. Vì đây cũng là ngôn ngữ được sử dụng cho
		\Gls{frontend} nên nhóm sẽ không phải tốn thêm thời gian tìm hiểu một ngôn ngữ khác cho phía \Gls{backend}
		\item Kiến trúc mà Nest.js đề xuất cũng tương đồng với kiến trúc của API Service mà nhóm đã thiết kế tại chương
		\ref{chap:system_design}, với các lớp controller, service, repository và cơ chế Dependency Injection tương tác
		giữa các lớp trên. Nhờ đó nhóm sẽ dễ dàng phát triển hệ thống được nhanh chóng hơn mà không phải suy nghĩ thêm về
		việc tổ chức cấu trúc dự án.
		\item Nest.js hỗ trợ tích hợp nhiều công cụ thuận tiện cho phát triển hệ thống, ví dụ như công cụ TypeORM giúp dễ
		dàng kết nói và ánh xạ các lớp model tới các bảng trong cơ sở dữ liệu và truy xuất dữ liệu từ đó.
		\item Nest.js có hệ thống tài liệu chỉn chu giúp dễ dàng tìm kiếm thông tin khi cần thiết. \Gls{framework} này cũng
		đang trở nên phổ biến hơn trong cộng đồng phát triển web nên sẽ dễ dàng hơn để nhóm tìm kiếm giải đáp.
	\end{itemize}

	\subsection{Lựa chọn hệ quản trị cơ sở dữ liệu (DBMS)}

	Đối với các hệ thống thông tin nói chung, hệ cơ sở dữ liệu là thành phần rất quan trọng và không thể thiếu. Đây là
	thành phần đóng vai trò lưu trữ, quản lý, truy vấn, cập nhật dữ liệu thực một cách tối ưu; cũng như các tính năng kiểm
	soát truy cập dữ liệu\ldots

	\subsubsection{PostgreSQL - hệ quản trị cơ sở dữ liệu}

	PostgreSQL là một hệ quản trị cơ sở dữ liệu quan hệ miễn phí và mã nguồn mở, với nhiều cải tiến mở rộng so với các hệ
	quản trị cơ sở dữ liệu truyền thống. PostgreSQL được phát triển bởi PostgreSQL Global Development Group và được phát
	hành lần đầu vào năm 1996, cho tới nay đã phát hành phiên bản 16.

	\begin{figure}[ht]
		\centering
		\includegraphics[scale=0.3]{images/techstacks/postgresql_logo}
		\caption{Logo của PostgreSQL}
		\label{fig:pg-logo}
	\end{figure}

	Nhóm chọn sử dụng PostgreSQL vì những lý do sau:

	\begin{itemize}
		\item PostgreSQL là hệ quản trị cơ sở dữ liệu quan hệ, phù hợp với phương hướng thiết kế của nhóm.
		\item PostgreSQL hỗ trợ nhiều dạng dữ liệu mà đặc biệt là JSON, giúp quá trình thiết kế cơ sở dữ liệu luận lý trở
		nên thuận tiện hơn.
		\item PostgreSQL có nhiều công cụ hỗ trợ tích hợp để có thể làm việc từ phía \Gls{backend} thay vì phải truy vấn
		trực tiếp trong database.
		\ item PostgreSQL có hệ thống tài liệu đầy đủ và cộng đồng hỗ trợ lớn mạnh.
	\end{itemize}

	\subsection{Một số công nghệ hỗ trợ}

	\subsubsection{Docker - Containerize ứng dụng}

	Docker là một công cụ hỗ trợ triển khai các ứng dụng vào trong các container ảo hoá. Docker được phát triển bởi công
	ty Docker.Inc từ năm 2013.

	\begin{figure}[ht]
		\centering
		\includegraphics[scale=0.3]{images/techstacks/docker_logo}
		\caption{Logo của Docker}
		\label{fig:docker-logo}
	\end{figure}

	Trong phát triển ứng dụng, khái niệm containerize được hiểu là ``đóng gói'' các ứng dụng trong những môi trường độc
	lập. Khái niệm này được đưa ra để giải quyết vấn đề cần phải triển khai hệ thống trên nhiều môi trường máy chủ khác
	nhau. Môi trường triển khai có nhiều yếu tố có thể ảnh hưởng tới khả năng hoạt động của hệ thống mà tiêu biểu là hệ
	điều hành, các thư viện, môi trường chạy ứng dụng (runtime environment), một số cấu hình (như cấu hình mạng, cấu hình
	biến môi trường\ldots) mà ứng dụng sẽ cần để có thể hoạt động trơn tru. Docker giúp tạo ra các môi trường ảo hoá dựa
	trên các bản phân phối hệ điều hành Linux và hỗ trợ người dùng cấu hình cho môi trường ảo hoá này trong file cấu hình
	là \emph{Dockerfile}. Dựa vào Dockerfile, khi triển khai ứng dụng trên một máy chủ mới, Docker sẽ luôn tạo ra chính
	xác một môi trường đúng với cấu hình mà nhà phát triển đã định nghĩa. Nhờ đó mà ứng dụng sẽ luôn hoạt động tốt trong
	môi trường ảo hoá này.

	Trong đề tài, nhóm dự kiến sẽ sử dụng Docker cho công cuộc kiểm thử tích hợp (integration test) và triển khai hệ
	thống.

	\newpage


	\section{Kế hoạch thực hiện giai đoạn Luận văn tốt nghiệp}

	Sau đây là bảng kế hoạch nhóm đề ra cho giai đoạn Luận văn tốt nghiệp với ước lượng thời gian khoảng 16 - 18 tuần.

	\begin{table}[ht]
		\centering
		\begin{tabular}{| c | p{10cm} | c |}
			\hline
			\textbf{Tuần} & \centering \textbf{Nội dung công việc}                                          
			& \textbf{Thành viên} 
			\\
			\hline

			\multirow{4}{*}{0} & Thời gian trống cho tới ngày bắt đầu chính th ức 
			& \multirow{4}{*}{Cả nhóm} 
			\\
			\cline{2-2}
			& Hoàn thiện các thiếu sót trong giai đoạn đề cương                               
			&                     
			\\
			\cline{2-2}
			& Chuẩn bị dữ liệu địa điểm và dữ liệu mẫu về xe                                  
			&                     
			\\
			\cline{2-2}
			& Khởi tạo môi trường hiện thực dự án                                             
			&                     
			\\
			\hline

			\multirow{2}{*}{1} & Hiện thực và kiểm thử UI trang đăng nhập, đăng ký và trang chủ người dùng
			& Nhật Tân
			\\
			\cline{2-3}
			& Hiện thực và kiểm thử API cho tính năng đăng nhập, đăng ký tài khoản, đăng xuất
			& Trọng Tuân
			\\
			\hline

			\multirow{2}{*}{2} & Hiện thực và kiểm thử UI trang hiển thị thông tin các xe, trang thông tin chi tiết của xe
			& Nhật Tân
			\\
			\cline{2-3}
			& Hiện thực và kiểm thử API đọc thông tin của các xe và thông tin chi tiết của xe
			& Trọng Tuân
			\\
			\hline

			\multirow{2}{*}{3} & Hiện thực và kiểm thử UI lọc thông tin tìm kiếm xe
			& Nhật Tân \\
			\cline{2-3}
			& Hiện thực và kiểm thử API lọc thông tin tìm kiếm xe
			& Trọng Tuân
			\\
			\hline

			\multirow{2}{*}{4} & Hiện thực và kiểm thử UI trang cập nhật thông tin người dùng, trang đăng ký cho thuê xe
			& Nhật Tân
			\\
			\cline{2-3}
			& Hiện thực và kiểm thử API cập nhật thông tin người dùng, API đăng ký cho thuê xe
			& Trọng Tuân
			\\
			\hline

			\multirow{2}{*}{5} & Hiện thực và kiểm thử UI tính năng đặt xe
			& Nhật Tân
			\\
			\cline{2-3}
			& Hiện thực và kiểm thử API tính năng đặt xe
			& Trọng Tuân
			\\
			\hline

			\multirow{2}{*}{6} & Hiện thực và kiểm thử UI tính năng đặt xe (thanh toán)
			& Nhật Tân
			\\
			\cline{2-3}
			& Hiện thực và kiểm thử API tính năng đặt xe (thanh toán)
			& Trọng Tuân
			\\
			\hline

			\multirow{2}{*}{7} & Hiện thực và kiểm thử UI tính năng huỷ đặt xe và nhắn tin
			& Nhật Tân
			\\
			\cline{2-3}
			& Hiện thực và kiểm thử API tính năng huỷ đặt xe
			& Trọng Tuân
			\\
			\hline

			\multirow{3}{*}{8} & Hiện thực và kiểm thử UI tính năng nhắn tin
			& Nhật Tân \\
			\cline{2-3}
			& Hiện thực và kiểm thử API tính năng nhắn tin
			& Trọng Tuân
			\\
			\cline{2-3}
			& Chuẩn bị báo cáo giữa kỳ
			& Cả nhóm
			\\
			\hline

		\end{tabular}
		\caption{\centering Bảng kế hoạch thực hiện giai đoạn Luận văn tốt nghiệp cho tới khi chấm giữa kỳ}
		\label{tab:thesis_plan}
	\end{table}

	\begin{table}[ht]
		\centering
		\begin{tabular}{| c | p{10cm} | c |}
			\hline
			\textbf{Tuần} & \centering \textbf{Nội dung công việc} & \textbf{Th
			ành viên} \\
			\hline

			\multirow{2}{*}{9} & Hiện thực và kiểm thử UI cho giao diện dashboard của officer
			& Nhật Tân
			\\
			\cline{2-3}
			& Hiện thực và kiểm thử API cho dashboard của officer
			& Trọng Tuân
			\\
			\hline

			\multirow{2}{*}{10} & Hiện thực và kiểm thử UI tính năng duyệt thông tin của officer
			& Nhật Tân
			\\
			\cline{2-3}
			& Hiện thực và kiểm thử API tính năng duyệt thông tin của officer
			& Trọng Tuân
			\\
			\hline

			\multirow{2}{*}{11} & Hiện thực và kiểm thử UI các bảng của officer
			& Nhật Tân
			\\
			\cline{2-3}
			& Hiện thực và kiểm thử API các bảng của officer
			& Trọng Tuân
			\\
			\hline

			\multirow{2}{*}{12} & Hiện thực và kiểm thử UI các bảng, tính năng cấm tài khoản của officer
			& Nhật Tân
			\\
			\cline{2-3}
			& Hiện thực và kiểm thử API các bảng, tính năng cấm tài khoản của officer
			& Trọng Tuân
			\\
			\hline

			\multirow{2}{*}{13} & Hiện thực và kiểm thử một số tính năng phụ như đa ngôn ngữ\ldots
			& Nhật Tân
			\\
			\cline{2-3}
			& Hiện thực và kiểm thử một số API tính năng phụ
			& Trọng Tuân
			\\
			\hline

			\multirow{3}{*}{14} & Kiểm thử integration test & \multirow{
				3}{*}{Cả nhóm} \\
			\cline{2-2}
			& Chuẩn bị dữ liệu thật
			&
			\\
			\cline{2-2}
			& Rà soát, sửa các lỗi xảy ra
			&
			\\
			\hline

			\multirow{3}{*}{15} & Kiểm thử integration test và stress test & \multirow{
				3}{*}{Cả nhóm} \\
			\cline{2-2}
			& Chuẩn bị dữ liệu thật
			&
			\\
			\cline{2-2}
			& Rà soát, sửa các lỗi xảy ra
			&
			\\
			\hline

			\multirow{3}{*}{16} & Quay dựng video demo
			& Nhật Tân
			\\
			\cline{2-3}
			& Thử triển khai ứng dụng lên hạ tầng thật
			& Trọng Tuân
			\\
			\cline{2-3}
			& Rà soát sửa các lỗi xảy ra
			& Cả nhóm
			\\
			\hline

			\multirow{3}{*}{17} & Tuần dữ trữ & \multirow{
				3}{*}{Cả nhóm} \\ \cline{2-2}
			& Kiểm thử, rà soát lỗi toàn bộ hệ thống
			&
			\\
			\cline{2-2}
			& Hoàn thiện báo cáo
			&
			\\
			\cline{2-2}
			\hline

			\multirow{3}{*}{18} & Tuần dữ trữ & \multirow{
				3}{*}{Cả nhóm} \\ \cline{2-2}
			& Kiểm thử, rà soát lỗi toàn bộ hệ thống
			&
			\\
			\cline{2-2}
			& Hoàn thiện báo cáo
			&
			\\
			\cline{2-2}
			\hline

		\end{tabular}
		\caption{\centering Bảng kế hoạch thực hiện giai đoạn Luận văn tốt nghiệp từ sau khi chấm giữa kỳ}
		\label{tab:thesis_plan_cont}
	\end{table}

	\newpage

\end{document}
