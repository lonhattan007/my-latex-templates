%! Compiler = lualatex --shell-escape
%! BibTeX Compiler = biber

\documentclass[../main.tex]{subfiles}
\graphicspath{{\subfix{../figures/}}}

\begin{document}

	\justifying


	\chapter{Thiết kế hệ thống}
	\label{chap:system_design}


	\section{Thiết kế kiến trúc hệ thống}

	Có nhiều mẫu kiến trúc được sử dụng cho các hệ thống thông tin như MVC, layered, client-server... Mẫu kiến trúc mà
	nhóm sử dụng là \emph{3-tier architecture} hay kiến trúc 3 tầng. Kiến trúc này cũng tương thích với kiến trúc phân lớp
	(layered) và kiến trúc máy chủ - máy khách (client - server). Hình \ref{fig:arch_diagram} biểu diễn kiến trúc cho hệ 
	thống.

	\begin{figure}[ht]
		\centering
		\includegraphics[scale=0.3]{images/system_design/arch_diagram.png}
		\caption{Thiết kế kiến trúc hệ thống}
		\label{fig:arch_diagram}
	\end{figure}

	Kiến trúc hệ thống bao gổm 3 tầng (tier) và giống với kiến trúc phân lớp, trong đó mỗi thành phần đảm nhận một nhiệm
	vụ riêng biệt và chỉ có thể giao tiếp với thành phần ở tầng kế cận. Vai trò của các tầng được thiết kế như sau:

	Tầng Presentation hay tầng biểu diễn là tầng đảm nhận vai trò hiển thị các giao diện người dùng để người dùng có thể
	nhìn thấy và tương tác được với hệ thống. Các thành phần giao diện lớn trong tầng này có thể chia ra là thành phần
	giao diện dành cho người dùng và thành phần dành giao diện dành cho officer. Khi triển khai, hai thành phần này có thể
	được triển khai trên hai domain khác nhau (ví dụ một domain và một subdomain), hoặc trên cùng một domain nhưng nằm ở
	khác đường dẫn.

	Tầng Application hay tầng ứng dụng là tầng đảm nhận các thông tin thu được từ tầng Presentation, sau đó là xử lý các
	logic sao cho đúng với business của hệ thống và tương tác với tầng Data để đảm bảo đọc và ghi đúng dữ liệu thực tế.
	Thành phần duy nhất trong tầng này là một dịch vụ API mà bản thân nó cũng được kiến trúc dạng phân lớp:

	\begin{itemize}
		\item Lớp Authentication and Authorization hay Xác thực và Phân quyền: đây là lớp đóng vai trò xác thực người dùng
		để xác định các thao tác hay đường dẫn mà người dùng đó được phép truy cập
		\item Lớp Controllers: đây là lớp đóng vai trò tiếp nhận các thao tác từ người dùng tới các URI tương ứng. Từ các
		URI và các yêu cầu thao tác \Gls{crud} của người dùng, nó sẽ gọi các phương thức phù hợp từ lớp Service và trả về
		các kết quả xử lý dữ liệu hoặc có thể là các lỗi xảy ra trong quá trình
		\item Lớp Services: đây là lớp chứa các logic chính của dịch vụ API. Dữ liệu từ Controllers gửi tới đây sẽ được xử
		lý và gọi xuống lớp Repositories để truy vấn các thông tin cần thiết. Kết quả này sau cùng sẽ trả về cho
		Controllers.
		\item Lớp Repositories: đây là lớp ánh xạ tới các bảng và các trường trong cơ sở dữ liệu tại tầng Data. Các thành
		phần trong lớp này cũng có các phương thức để truy vấn hay cập nhật dữ liệu từ các bảng tương ứng trong cơ sở dữ
		liệu.
	\end{itemize}

	Tầng Data hay tầng dữ liệu là nơi chứa cơ sở dữ liệu để cho tầng Application có thể truy xuất các dữ liệu đó. Trong 
	thiết kế thiến trúc hiện tại của nhóm, hệ thống sẽ sử dụng một cơ sở dữ liệu tập trung vì lượng dữ liệu ban đầu sẽ 
	chưa đủ lớn để dùng tới các chiến lược phân tán cơ sở dữ liệu.

	Khi kết hợp với các công nghệ sẽ được trình bày tại chương \ref{chap:implementation}, nhóm đã vẽ ra thiết kế kiến trúc
	của hệ thống khi được triển khai tại hình \ref{fig:deploy_diagram}:

	\begin{figure}[ht]
		\centering
		\includegraphics[scale=0.3]{images/system_design/deploy_diagram.png}
		\caption{Thiết kế triển khai hệ thống}
		\label{fig:deploy_diagram}
	\end{figure}

	Có thể thấy mỗi thành phần trong thiết kế trên tương ứng với mỗi tầng trong kiến trúc đã đề ra.


% \section{Mô hình hoá hệ thống}

% \subsection{Class diagram}

% \subsection{Sequence diagram}

% \subsection{Activity diagram}

% \subsection{EERD}

% \section{Thiết kế giao diện ứng dụng}

	\newpage

\end{document}
